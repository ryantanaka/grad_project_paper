\documentclass{article} \usepackage{indentfirst}

\begin{document}

\title{WRENCH Pedagogic Modules: Distributed Computing Courseware} \author{Ryan
Tanaka}

\maketitle

\section{Introduction} The WRENCH Pedagogic Modules are a guided set of hands
on, educational activities aimed at teaching distributed computing concepts
using the WRENCH workflow management system simulation framework introduced in
\cite{wrench}. This simulation based, distributed computing courseware was
developed with the goal of accelerating distributed computing education in the
classroom by affording higher student participation over traditional methods and
by providing a more effective means of pedagogy. To achieve this goal, we
utilize the WRENCH simulation framework along with contemporary web
technologies. These modules are not meant to replace a conventional education on
distributed computing, rather they are meant to supplement it so as to deliver a
modern, yet effective educational  experience.

This paper is organized as follows. First I elaborate on the benefits of using
simulation for teaching  distributed computing. Second, the design and layout of
these modules are described. Next, I discuss results we have obtained from two
initial studies. Finally, I detail future plans to build upon the current set of
modules and also to submit our work to a high performance computing education
conference.

\section{Using Simulations} Conventional distributed computing education often
consists of theory incorporated with a mix of hands-on learning experiences
using real world platforms. While this is a widely accepted method of teaching
distributed computing, it has been shown that both \textit{logistical
challenges} and \textit{pedagogical challenges} stifle its effectiveness
\cite{smpi-courseware}. Logistical challenges arise when students need access to
cyber infrastructure platforms on which they can execute hands-on experiments.
Not all institutions have such access to various  cyber infrastructure systems.
Furthermore, access to these systems is limited due to quantity and availability
constraints. Assuming access to one type of system is given, pedagogical
challenges are still present. First, training or "on boarding" is necessary to
get students familiarized with that system. Second, students are limited to
predetermined cyber infrastructure configurations, therefore limiting their
ability to experiment with "What if?" scenarios. For example, it will not be
possible for students to  run hands on experiments on a wide variety of cyber
infrastructure scenarios without access to multiple systems due to the
aforementioned  logistical challenges.

With the use of WRENCH simulation software, arbitrary cyber infrastructure can
be simulated on a single commodity computer, therefore directly addressing both
logistical and  pedagogical challenges. Student participation only requires a
desktop or laptop with modest hardware specifications. Executing WRENCH
simulations requires basic knowledge of how to run programs from a command line
and time spent "on boarding" in this case is negligible. In regards to pedagogy,
this simulation software affords students the ability to experiment with "What
if?" scenarios, thereby enriching their understanding of the theoretical
concepts at hand. The underlying simulation engine used in WRENCH is provided by
the SimGrid distributed systems simulation framework \cite{simgrid}, whose
simulation models have been  rigorously validated over the years.

We are confident that with the use of such accurate simulation software along
with a web based, guided curriculum, a valuable contribution can be made to
distributed computing pedagogy. Next, I introduce the WRENCH Pedagogic Modules.

\section{The WRENCH Pedagogic Modules} The WRENCH Pedagogic Modules are a guided
set of hands on, educational activities focused on teaching distributed
computing concepts through the use of interactive simulations. It currently
contains five guided lessons hosted online, four WRENCH simulations, and an
interactive browser-based dashboard from which users can run those simulations
and view results through a number of different custom visualizations. The target
audience for these modules are computer science undergraduate and graduate
students interested in distributed computing. While previous knowledge of
distributed computing concepts is helpful, it is not required to make the most
out of these modules as ample background information is provided throughout.

Each of the five lessons contain student learning objectives, background
information on the specific distributed computing concept being covered, and a
handful of questions that lead students toward achieving each learning objective
in a moment of revelation. Many of these questions require students to formulate
a simple mathematical model based on some described scenario, then have them
check that model against results returned from running the simulation through
the dashboard. Then the scenario is changed so that students can be exposed to
the "What if?" scenarios. For example,  a series of performance related
questions might be asked regarding a high performance computing cluster given
some algorithm and workload. First, students are asked to investigate this
question assuming the cluster has a single compute node. Then, follow up
questions are asked that modify the scenario such that the cluster now has
multiple compute nodes. In both scenarios, students can verify their results by
running the simulations through our dashboard.

In order to address the issue of dependencies, we host these modules online at
\underline{https://wrench-project.org/wrench-pedagogic-modules/} and
containerize the dashboard along with each of the simulations using Docker. As a
result, students need only a modest computer, an internet connection, a web
browser, and Docker or Vagrant installed to participate.

\section{Preliminary Results} The efficacy of these modules is currently being
evaluated based on feedback and results obtained from trials: a focus group
activity and an in class implementation. The focus group consisted of  3
undergraduate computer science students who had completed an operating systems
course  prior to participating. For the in class implementation, a portion of
these modules were adopted into the Spring 2019 operating systems undergraduate
course which consisted of 59 students.

Two methods of data collection were employed in both trials. First,
pre-knowledge and post-knowledge written tests were given to students before and
after completing a module. Results from these tests are currently being
evaluated. Second, usage metrics were collected through the dashboard and stored
in a database. From these metrics, we can determine who ran what simulation,
with what parameters, and at what time. A brief preliminary analysis shows that 


\section{Future Work}

\bibliographystyle{ieeetr} \bibliography{sources}

\end{document}

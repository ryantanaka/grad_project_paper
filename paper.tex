\documentclass{article}
\usepackage{indentfirst}

\begin{document}

\title{WRENCH Pedagogic Modules: Distributed Computing Courseware}
\author{Ryan Tanaka}

\maketitle

\section{Introduction}
The WRENCH Pedagogic Modules are a guided set of hands on, educational
activities aimed at teaching distributed computing concepts using the
WRENCH workflow management system simulation framework introduced in \cite{wrench}. This simulation
based, distributed computing courseware was developed with the goal
of accelerating distributed computing education in the classroom by affording
higher student participation over traditional methods and by providing a
more effective means of pedagogy. To achieve this goal, we utilize the WRENCH
simulation framework along with contemporary web technologies. These modules are 
not meant to replace a conventional education on distributed computing, rather 
they are meant to supplement it so as to deliver a modern, yet effective educational 
experience. 


This paper is organized as follows. First I elaborate on the benefits of using simulation for teaching 
distributed computing. Second, the design and layout of these modules are described. Next, I discuss
results we have obtained from two initial studies. Finally, I 


\section{Using Simulations}

\section{The WRENCH Pedagogic Modules}

\section{Preliminary Results}

\section{Future Work}

\bibliographystyle{ieeetr}
\bibliography{sources}

\end{document}
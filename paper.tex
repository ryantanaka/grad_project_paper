\documentclass{article}
\usepackage{indentfirst}

\begin{document}

\title{WRENCH Pedagogic Modules: Distributed Computing Courseware}
\author{Ryan Tanaka}

\maketitle

\section{Introduction}
The WRENCH Pedagogic Modules are a guided set of hands on, educational
activities aimed at teaching distributed computing concepts using the WRENCH
workflow management system simulation framework introduced in \cite{wrench}.
This simulation based, distributed computing courseware was developed with the
goal of accelerating distributed computing education in the classroom by
affording higher student participation over traditional methods and by providing
a more effective means of pedagogy. To achieve this goal, we utilize the WRENCH
simulation framework along with contemporary web technologies. These modules are
not meant to replace a conventional education on distributed computing, rather
they are meant to supplement it so as to deliver a modern, yet effective
educational  experience.

This paper is organized as follows. First I elaborate on the benefits of using
simulation for teaching  distributed computing. Second, the design and layout of
these modules are described. Next, I discuss results we have obtained from two
initial studies. Finally, I detail future plans to build up the current set of
modules and also to submit our work to a high performance computing education
conference.

\section{Using Simulations}
Conventional distributed computing education often consists of theory
incorporated with a mix of hands-on learning experiences with real world
platforms. While this is a widely accepted method of teaching distributed computing,
it has been shown that both \textit{logistical challenges} and \textit{pedagogical challenges}
stifle its effectiveness \cite{smpi-courseware}. Logistical challenges arise when
students need access to cyber infrastructure platforms on which they can execute
hands-on experiments. Not all institutions have such access to various 
cyber infrastructure systems. Furthermore, access to these systems is limited due
to quantity and availability constraints. Assuming access to one type of system is
given, pedagogical challenges are still present. First, training or "on boarding" is necessary
to get students familiarized with that system. Second, students are limited to 
predetermined cyber infrastructure configurations, therefore limiting their ability to
experiment with "What if?" scenarios. For example, it will not be possible for students to 
run hands on experiments on a wide variety of cyber infrastructure scenarios without
access to multiple systems, which is likely not possible due to aforementioned 
logistical challenges. 

With the use of WRENCH simulation software, arbitrary cyber infrastructure can be simulated
on a single commodity computer, therefore directly addressing both logistical and 
pedagogical challenges. Student participation only requires a desktop or laptop with modest
hardware specifications. Executing WRENCH simulations require basic knowledge of how to
run programs from a command line. As such, time spent "on boarding" in this case is
negligible. 

\section{The WRENCH Pedagogic Modules}

\section{Preliminary Results}

\section{Future Work}

\bibliographystyle{ieeetr}
\bibliography{sources}

\end{document}
